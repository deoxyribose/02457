\documentclass[12pt]{article}    % Specifies the document style.
\usepackage{amsmath}

\textwidth 16cm
\textheight 24cm
\oddsidemargin 0cm
\topmargin -1cm

\def\xb{{\bf x}}
\def\zb{{\bf z}}
\def\wb{{\bf w}}
\def\Ac{\mathcal{A}}
\def\Dc{\mathcal{D}}
\def\mub{\text{\boldmath $\mu$}}
\def\Sigb{\text{\boldmath $\Sigma$}}
\def\Sb{\text{\boldmath $S$}}
\def\L{\text{\boldmath $\Lambda$}}
\def\Ub{{\bf U}}
\def\vb{{\bf v}}
\def\ub{{\bf u}}
\def\db{{\bf d}}
\def\gb{{\bf g}}
\def\Hb{{\bf H}}
\def\Wb{{\bf W}}
\def\squeeze{\itemsep=0pt\parskip=0pt}

\begin{document}
\section*{{\it COURSE 02457}\\[5mm] Non-Linear Signal Processing: Exercise 8}



This exercise is based on in C. M. Bishop: {\em Pattern
Recognition and Machine Learning} section~6.3.

Print and comment on the figures produced by the software as
outlined below at the {\bf Checkpoints}.

\subsection*{Regression based on density estimation}



We observe a stochastic multi-channel signal with $d$-dimensional
inputs $\xb$ and scalar outputs $t$ and our aim is model the
density $p(\xb,t) \sim p(\xb,t|\wb)$ where the family
$p(\xb,t|\wb)$ is a given parametric density. The training set is
consists of N input-output pairs.

The mixture of Gaussians model family is defined
\begin{eqnarray}
p(\xb,t|\wb) = \sum_{j=1}^{M} p(\xb,t|\wb_j)P(j) \ , \nonumber
\end{eqnarray}
where each of the $M$ component densities is a normal
distribution with parameters $\wb_j = \left\{\mub_j, \sigma^{2}_j  \right\}$. $P(j)$ is the mixing proportion for component $j$,
that is the prior probability that the data comes from this
component. Here we will invoke a family of ``isotropic''
Gaussians, i.e., Gaussians with covariance matrices that are
scaled unit matrices,
\begin{eqnarray}
p(\xb,t|\mub_j,\sigma^{2}_j) &=&
\frac{1}{(2\pi\sigma^{2}_{j})^{(d+1)/2}} \exp\left(-\frac{||\xb
-\mub_j||^2 + (t - \nu_j)^2}{2\sigma^{2}_j}\right). \nonumber
\end{eqnarray}
$\mub_j$ is mean vector in input space and $\nu_j$ the mean in the
output space. This model makes the important simplifying
assumption that within each component, the variance is identical
in all directions including the output direction.

We will estimate the parameters using maximum likelihood as implemented by
the Expectation-Maximization
algorithm as in exercise 7 (see also Bishop chapter 9). The cost function is therefore
$$
E = - \sum_{n=1}^N \log p(\xb_n,t_n|\wb) \ .
$$
%
Based on the joint input-output distribution we can compute the conditional distribution
\begin{eqnarray*}
p(t|\xb) & = & \frac{p(\xb,t) }{p(\xb) }  \\ p(\xb) & = & \int p(\xb,t) \, dt = 
\sum_{j'=1}^{M} \frac{P(j')}{
(2\pi\sigma^{2}_{j'})^{\frac{d}{2}}} \exp \left(
-\frac{||\xb-\mub_{j'}||^2}{2\sigma^{2}_{j'}}\right) \ , 
\end{eqnarray*}
where the explicit expression for $p(\xb)$ obtained by marginalization over $t$ is a standard Gaussian integral (see Bishop Section 2.3).  
The conditional output mean (Bishop section 6.3) is the optimal regression function for least squares loss function:
\begin{eqnarray*}
y(\xb) & = & <t|\xb> \equiv \int \, t \, p(t|\xb) dt \ .
\end{eqnarray*}
The explicit expression for the conditional mean is
\begin{eqnarray}
y(\xb) = \frac{\sum_{j=1}^{M} \, \nu_j \,
\frac{P(j)}{(2\pi\sigma^{2}_{j})^{\frac{d}{2}}} \exp
\left(-\frac{||\xb-\mub_j||^2}{2\sigma^{2}_{j}}\right)}
{\sum_{j'=1}^{M} \frac{P(j')}{
(2\pi\sigma^{2}_{j'})^{\frac{d}{2}}} \exp \left(
-\frac{||\xb-\mub_{j'}||^2}{2\sigma^{2}_{j'}}\right)}\nonumber
%y(\xb) = \frac{\sum_{j=1}^{M} P(j) \nu_j \exp \left( -\frac{(\xb-\mub_j)^2}{2\sigma^{2}_{j}}\right)}
%{\sum_{j=1}^{M} P(j) \exp \left( -\frac{(\xb-\mub_j)^2}{2\sigma^{2}_{j}}\right)}\nonumber
\end{eqnarray}
We see that this expression contains the responsibility
$$
P(j|\xb) = \frac{\frac{P(j)}{(2\pi\sigma^{2}_{j})^{\frac{d}{2}}} \exp
\left(-\frac{||\xb-\mub_j||^2}{2\sigma^{2}_{j}}\right)}
{\sum_{j'=1}^{M} \frac{P(j')}{
(2\pi\sigma^{2}_{j'})^{\frac{d}{2}}} \exp \left(
-\frac{||\xb-\mub_{j'}||^2}{2\sigma^{2}_{j'}}\right)} \ .
$$
The conditional mean is therefore a convex combination of the components
means in output space:
$$
y(\xb) = \sum_{j=1}^{M} \, \nu_j \, P(j|\xb) \ .
$$

\subsubsection*{Checkpoint 8.1}

Use the program {\tt main8a.m} to perform EM learning of the
parameters for the sunspot data. The program creates four figures.
The first figure is a scatter plot of training points $x(k)$
versus $y(k) =x(k+1)$, and also the location of the centers as
iterations progress. The second plot shows the evolution of the
variances $\sigma_j$. By default we clamp them to be identical by
setting {\tt common-sigs = 1}. The second figure also shows the
evolution of the training and test errors of the density
estimates. The third figure is like figure 1, but with the final
clusters. Figure 4 shows the training and test set time series.
The program also reports (in the Matlab prompt line) the training
and test errors estimated by prediction. These are calculated as
in earlier exercises by the normalized squared error of the
predictions on training and test sets. In the program you can
also change the number of mixture components $K$. First run the
program with $K=5,25$ with common covariances. Do you see
overfitting in the density test error (figure 2)? Do you see overfitting in the
squared error? Set {\tt
common-sigs = 0} and comment on the results for $K = 5,25$. If you
experience problems with shrinking variances, try to introduce a
lower bound, what is a good value? Compare the results obtained
with those of multilayer perceptron nets from earlier exercises.


\subsubsection*{Checkpoint 8.2}

To run the program {\tt main8a.m} with higher $K$ values you may
want to eliminate intermediate plots by setting {\tt
plot-motion=0}. Run the algorithm for higher $K$'s to find the
best $K$ from a prediction point of view.


\subsection*{Signal detection based on density estimation}

If we adapt density models $P(\xb|C_k)$ for each class of a
classification problem we can use Bayes' theorem to obtain the
posterior probabilities (see Bishop section 1.5 and this week's
lecture slide),
\begin{eqnarray}
P(C_k|\xb) & =&  \frac{\sum_{j}p(\xb|j) P(j|C_{k})P(C_k)}{\sum_{j'}p(\xb|j')P(j')} \ .
 \nonumber
\end{eqnarray}
In the above expression we use a mixture of Gaussians to model the density, i.e.\ $p(\xb|j)$
is a normal and $P(j|C_{k})$ is probability (table) indicating which components belong to
which classes.

\subsubsection*{Checkpoint 8.3}

Use the program {\tt main8b.m} to adapt densities for the two
classes of the ``Pima indian problem''. In figure 1 we show
scatter plot of two of the seven pima inputs and superimpose the
motion of the centers as we adapt the densities. This program also
reports two test errors: a density estimate test error (figure 2)
and the classification test error (prompt line). Figure 3 shows
the location of the components for the final configuration. Create
a flowchart description of the program. Run the program for
different $K$, comment on the location of the components. Which
value of $K$ can you recommend?

\subsubsection*{Challenge I (not part of the curriculum)}

Use the results of Challenge I in Exercise 7 to run the Checkpoints in this exercise with full covariance matrices instead of isotropic covariances. Do you see improvements over the isotropic model results? 


\vspace{2cm} DTU, 2006 Lars Kai Hansen (2007, 2009 Ole Winther)


\end{document}
